\chapter{A New Semantics for Synchronized P Systems}

For the purpose of this thesis it is useful to define a new semantics for our synchronized P systems.
We know that a rule $r$ that compose a synchronization rule $r_s$ can be executed if and only $r_s$ it's executed during the same evolution step. 
So we could say that the application of the rule $r$ causally depends on the application of the synchronization rule $r_s$.\newline

Let's say that an evolution step is an ordered sequence of micro evolution steps obtained by the application of a single evolution rule.
So now a classical maximal parallelism evolution step is represented as a sequence of simple evolution steps.
This is called \textit{causal semantics}, and was first defined for a basic P system in \cite{busi2007causality}.

\begin{definition}[Configuration and partial configuration]
    In our new semantics a configuration is a pair $C=(w_1,w_2)$ where $w_1$ is the active multiset and $w_2$ is the deposit (or frozen) multiset.\newline
    We call a configuration $C=(w_1,w_2)$ where $w_2 \neq \emptyset$ a \textit{partial configuration}; if $w_2 = \emptyset$ we'll refer to $C$ as a configuration.
\end{definition}

So now the initial configuration of a P system $\Pi$ is $C_0=(w_1,\emptyset)$.
Doing a whole evolution step means to go from a configuration $(w_1,\emptyset)$ to another configuration $(w_1',\emptyset)$ passing through a sequence of micro steps.

Before providing a new definitions of an evolution step and a micro step, we need a some auxiliary definitions.
We first introduce the concept of \textit{causal configuration}, then we introduce a new function, the \textit{transport function}, that transports the objects in the deposit multiset to the active multiset; it can be seen as a special evolution rule proper of the semantics itself.

\begin{definition}[Causal configuration]
    A causal configuration is a tuple 
    \[CC = \langle C, \Gamma \rangle\]
    where $C$ is a configuration (or a partial configuration) and $\Gamma$ is a set of synchronization rules.
    Given an initial configuration $C_0$ the respective causal configuration is $CC = \langle C_0,\emptyset\rangle$.
\end{definition}

\begin{definition}[transport function]
    Let $C=(w,w')$ be a partial configuration of $\Pi$.\newline
    The transport function $\delta: Conf_\Pi \rightarrow Conf_\Pi$ is defined as follows:\newline
    $\delta((w,w'))=(w \oplus w', \emptyset)$
\end{definition}

With the notion of transport function and causal configuration we can now define what a computational step and a micro step are.

\begin{definition}[computational step and micro steps]
    A computational step is defined as an ordered sequence of micro steps.\newline
    $(w_1,\emptyset) \Rightarrow (w_1',\emptyset)$ is a computational step if and only if 
    there exists $k$ micro steps such that $(w_1,\emptyset) \xrightarrow{r_1} (w_1'',w_2'')
    \xrightarrow{r_2} \cdots \xrightarrow{r_k} (w_1',\emptyset)$ where $r_k$ is the transport function $\delta$.
    We also define $r=<r_1,\ldots,r_k>$ as the ordered sequence of evolution rules applied at each micro step.
\end{definition}

Now we describe how our semantics works with the following inference rules:\newline

\noindent
\begin{tabular}{ @{} r c l @{} }
  R1. & \infer{
  \langle(w,w'),\Gamma\rangle \xrightarrow{\delta} \langle(w \oplus w', \emptyset),\emptyset\rangle
  }
{\text{no rule is enabled}} \\ \\
  R2. & \infer{
  \langle(w,w'),\Gamma\rangle \xrightarrow{r_s} \langle(w - lhs(r_s), w' + rhs(r_s)),\Gamma \cup 
  \{r_s\} \rangle 
  }
{\text{$r_s$ is enabled in $(w,w')$}} \\ \\
  R3. & \infer{
  \langle (w,w'), \Gamma \rangle \xrightarrow{r} \langle (w - lhs(r_s), w' + rhs(r_s)), \Gamma \rangle 
  }
{\text{$r$ is enabled in $(w,w')$ and if $r$ compose $r_s$, $r_s \in \Gamma$}} \\ \\
\end{tabular}

Let's see an example of our new semantics.
\begin{example}
    Consider the system 
    \[ \Pi = (\{a,b,c,d\},[_1]_1,a^2bc,\{r_1:a \rightarrow b, r_2: b \rightarrow c\}, \{r_{12}: r_1 r_2\}) \]
    The only rule we can apply is R2, since $r_{12}$ is enabled in $(a^2bc,\emptyset)$ and no other evolution rule is.
    After this micro step we have $C = (ac,bc)$.
    Now we can apply rule R3, so we apply the evolution rule $r_1$ thus obtaining the partial configuration $C = (c,b^2c)$.
    No other evolution rule is enabled, so R1 is the only rule applicable, and we get 
    $C = (b^2c^2)$, and this is the configuration obtained after a whole evolution step.
\end{example}