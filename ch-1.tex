\chapter{Introduction}

Membrane computing is a branch of natural computing, a field of research which tries to imitate nature in the ways it computes.
The obtained computing models are distributed parallel devices, called P systems (known also as Membrane systems).
The main aim of membrane computing is to model parallel systems inspired by the structure and behaviour of biological cells, but P systems are also rather powerful, equivalent with Turing machines.\\

Petri nets are an abstract, formal model of information flow.
The major use of Petri nets is to model concurrent systems, and P systems can be seen as highly concurrent systems, indeed in this thesis we try (successfully) to model the behaviour of a particular class of membrane systems, membrane systems which use synchronization among the rules of the same membrane, that we're gonna just call \textit{synchronization P systems}, using extended Petri nets with inhibitor arcs. \\

In the next chapter we we will elaborate and formally define what P systems, Petri nets and their extensions are.
In the third chapter we first establish a link between a basic class of P systems and Petri nets, then between synchronization P systems and Petri nets with inhibitor arcs.
Chapter 4 provides the proof that computations of a synchronized P systems coincide with step sequences of the relative extended Petri net. 
Finally, in chapter 5 we talk about future developments and changes to adopt.